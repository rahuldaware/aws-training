\documentclass{article}
\usepackage{hyperref}
\usepackage{color}
\addtolength{\oddsidemargin}{-.875in}
\addtolength{\evensidemargin}{-.875in}
\addtolength{\textwidth}{1.75in}
\addtolength{\topmargin}{-.875in}
\addtolength{\textheight}{1.75in}
\pagestyle{plain}
\author{Rahul Daware}
\begin{document}
\title{Databases on AWS}
\maketitle
\newpage
\tableofcontents
\newpage

\section{Relational Databases}
Amazon offers following relational dabases:

\begin{itemize}
\item
SQL Server

\item
Oracle

\item
MySQL Server

\item
PostgreSQL

\item
Aurora

\item
MariaDB
\end{itemize}

RDS has two key features:
\begin{itemize}
\item
Multi AZ - For Disaster Recovery

\item
Read Replicas - For Performance
\end{itemize}

Remember the following points:
\begin{itemize}
\item
RDS runs on virtual machines

\item
You cannot log in to these operating systems however

\item
Patching of the RDS Operating System and DB is Amazon's responsibility

\item
RDS is NOT serverless

\item
Aurora serverless is serverless

\end{itemize}

\section{Backups with RDS}
There are two different types of backups for RDS:
\begin{itemize}
\item
Automated Backups

\item
Database Snapshots
\end{itemize}
\subsection{Automated Backups}
Automated Backups allow you to recover your database to any point in time within a "retention period". The retention period can be between one and 35 days. Automated backups will take a full daily snapshot and will also store transaction logs throughout the day. When you do a recovery, AWS will first choose the most recent daily back up,and then apply transaction logs relevant to that day. This allows you to do a point in time recovery down to a second, within the retention period.

Automated backups are enabled by default. The backup data is stored in S3 and you get free storage space equal to the size of your database. So if you have an RDS instance of 10GB, you will get 10GB worth of storage. Backups are taken within a defined window. During the backup window, storage I/O may be suspended whileyour data is being backed up and you may experience elevated latency.

\subsection{Database Snapshots}
DB Snapshots are done manually (i.e. they are user initiated) They are stored even after you delete the original RDS instance, unlike automated backups.

Whenever you restore either an automated backup or a manual snapshot, the restored version of the data base will be a new RDS instance with a new DNS endpoint.

\subsection{Encryption at Rest}
Encryption at rest is supported for MySQL, Oracle, SQL Server, PostgreSQL, MariaDB and Aurora. Encryption is done using the AWS key management service (KMS) service. Once your RDS instance is encrypted, the data stored at rest in the underlying storage is encrypted, as are its automated backups, read replicas, and snapshots.

\subsection{What is Multi AZ}
Multi-AZ allows you to have an exact copy of your production database in another availability zone. AWS handles the replication for you, so when your production database is written to, this write will automatically be synchronized to the stand by database.

In the event of planned databae maintenance, DB instance failure, or an availabilit failover to the standby so that database operations can resume quickly without administrative intervention.

\subsection{Read Replica}
Read Replica allow you to have a read-only copy of your production database. This is achieved by using asynchronous replication from the primary RDS instance to the read replica. You use read replicas primarily for very read-heavy database workloads.

Read replicas are available for the following databases:
\begin{itemize}
\item
MySQL server

\item
PostgreSQL

\item
MariaDB

\item
Aurora
\end{itemize}

Things to know about read replicas:
\begin{itemize}
\item
Used for scaling, not for DR

\item
Must have automatic backups turned on in order to deploy a read replica

\item
You can have up to 5 read replica copies of any database

\item
You can have read replicas of read replicas. (but watch out for latency)

\item
Each read replica will have its own DNS endpoint

\item
You can have read replicas that have multi AZ

\item
You can create read replicas of multi AZ source databases

\item
Read replicas can be promoted to be their own databases. This breaks the replication.

\item
You can have a read replica in a second region.
\end{itemize}

\end{document}