\documentclass{article}
\usepackage{hyperref}
\usepackage{color}
\addtolength{\oddsidemargin}{-.875in}
\addtolength{\evensidemargin}{-.875in}
\addtolength{\textwidth}{1.75in}
\addtolength{\topmargin}{-.875in}
\addtolength{\textheight}{1.75in}
\pagestyle{plain}
\author{Rahul Daware}
\begin{document}
\title{AWS 10000 Foot Overview}
\maketitle
\newpage
\tableofcontents
\newpage

\section{Highlights}
\begin{itemize}
\item
Edge locations are endpoints for AWS which are used for caching content. Typically this consists of CloudFront, Amazon's Content Delivery Network(CDN). There are many more edge locations than regions. Currently, there are over 150 edge locations.

\item
A region is a physical location in the world which consists of two or more availability zones.

\item
An AZ is one or more discrete data centers, each with redundant power, networking and connectivity, housed in seperate facilities.

\item
Edge locations are endpoints  for AWS which are used for caching content. Typically this consists of CloudFront, Amazon's Content Delivery Network (CDN).

\item
High level services which you should focus (Core services in bold): 
	\begin{itemize}
	
	\item
	\textbf{Compute}
	
	\item
	\textbf{Storage}
	
	\item
	\textbf{Databases}
	
	\item
	Migration and Transfer
	
	\item
	\textbf{Network and Content Delivery}
	
	\item
	Management and Governance
	
	\item
	Machine Learning
	
	\item
	Analytics
	
	\item
	\textbf{Security, Identitiy and Compliance}
	
	\item
	Desktop and App Streaming
	
	\end{itemize}
\end{itemize}
\end{document}