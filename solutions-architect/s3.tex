\documentclass{article}
\usepackage{hyperref}
\usepackage{color}
\addtolength{\oddsidemargin}{-.875in}
\addtolength{\evensidemargin}{-.875in}
\addtolength{\textwidth}{1.75in}
\addtolength{\topmargin}{-.875in}
\addtolength{\textheight}{1.75in}
\pagestyle{plain}
\author{Rahul Daware}
\begin{document}
\title{Chapter 4: Simple Storage Service (S3)}
\maketitle
\tableofcontents
\newpage

\section{What is S3?}
S3 provides developers and IT teams with secure, durable, highly scalable object storage. Amazon S3 is easy to use, with a simple web services interface to store and retrieve any amount of data from anywhere on the web. S3 is a safe place to store your files. It is object based storage. The date is spread across multiple devices and facilities.

\section{S3 - The Basics}
\begin{itemize}
\item
S3 is object based - i.e. allows you to upload files

\item
Files can be from 0 bytes to 5 TB.

\item
There is unlimited storage.

\item
Files are stored in buckets. (Buckets are folder in cloud. Each bucket has universal namespace. Each bucket will have a DNS address. Bucket names have to unique.

\item
S3 is a universal namespace. i.e. names must be unique globally.

\item
URL look like this : https://s3-eu-west-1.amazonaws.com/acloudguru

\item
When you upload a file to S3, you will receive a HTTP 200 if the upload was successful

\item
Built for 99.99\% availability for the S3 platform

\item
Amazon guarantee 99.9\% availability

\item
Amazon guarantees 11 9s durability for S3 information

\item
Tiered Storage available

\item
Lifecycle Management

\item
Versioning

\item
Encryption

\item
Secure your data using Access Control Lists and Bucket Policies

\end{itemize}

\section{Data Consistency Model for S3}
\begin{itemize}
\item
Read after write consistency for PUTS of new objects

\item
Eventual consistency for overwrite PUTS and DELETES (can take some time to propogate)
\end{itemize}

\section{S3 is a simple key-value store}
S3 is object based. Objects consists of the following:
\begin{itemize}
\item
Key (This is simply name of the object
\item
Value ((This is simply the data and is made up of a sequence of bytes
\item
Version ID (Important for versioning)
\item
Metadata (Data about data you are storing)
\item
Subresources
	\begin{itemize}
	\item
	Access Control Lists
	
	\item
	Torrent
	
	\end{itemize}
\end{itemize}

\section{S3 - Storage Tiers/Classes}
\begin{itemize}
\item
\textbf{S3 Standard} : 99.99\% availability 11 9s durability, stored redundantly across multiple facilities, and is designed to sustain the loss of 2 facilities concurrently.

\item
\textbf{S3 -IA} : (Infrequently Accessed): For data that is accessed less frequently, but requires rapid access when needed. Lower fee than S3, but you are charged a retrieval fee.

\item
\textbf{S3 One Zone - IA} Want a lower cost option for infrequently accessed data, but do not require the multiple availability zone data resilience.

\item
\textbf{Glacier} Very cheap, but used for archival only. Expedited, standard or bulk. A standard retrievel time takes 3-5 hours.
\end{itemize}

\section{S3 - Charges}
Charged for:
\begin{itemize}
\item
Storage (Per GB basis)

\item
Requests (No. of requests)

\item
Storage Management Pricing ( Charged for tags on data)

\item
Data Transfer Pricing (Transferring data from one region to other)

\item
Transfer Acceleration : Amazon S3 Transfer acceleration enables fast, easy and secure transfers of files over long distances between your end users and an S3 bucket. Transfer acceleration takes advantage of Amazon CloudFront's globally distributed edge locations. AS the data arrives at an edge location, data is routed to Amazon S3 over an optimized network path.
\end{itemize}

\section{S3 Exam Tips for S3 101}
\begin{itemize}
\item
Remember that S3 is object based. i.e. allows you to upload files

\item
Files can be from 0 bytes to 5 TB

\item
There is unlimited storage.

\item
Files are stored in buckets

\item
S3 is a universal namespace. That is names must be unique globally

\item
Read afte write consistency for PUTS of new objects

\item
Eventual Consistency for overwrite PUTS and DELETES (can take some time to propogate)

\item
Storage Classes/Tiers:
	\begin{itemize}
	\item
	S3 (durable, immediately available, frequently accessed)
	
	\item
	S3 -IA (Durable, immediately available, infrequently accessed)
	
	\item
	S3 One Zone - IA (event chaeaper than IA, but onle in one AZ)
	
	\item
	Glacier - Achived data, where you cn wait 3-5hours before accessing
	\end{itemize}

\item
Remember core fundamentals of S3 object - Key (name), Value (data) , Version ID, Metadata, Subresources (ACL, Torrent)

\item
Object based storage only (for files)

\item
Not suitable to install an operating system on

\item
Successful uploads give HTTP 200 response

\end{itemize}

\section{Create an S3 Bucket - Exam Tips}
\begin{itemize}
\item
Buckets are a univeral name space

\item
Upload an object to S3 receive a HTTP 200 code

\item
S3, S3 -IA, S3 - One Zone IA, S3 Reduced Redundancy Storage

\item
Encryption
	\begin{itemize}
	\item
	Client Side Encryption
	
	\item
	Server Side Encryption with Amazon S3 Managed Keys (SSE-S3), Server Side Encryption with KMS (SSE-KMS), Server Side Encryption with Customer provided keys (SSE-C)
	\end{itemize}

\item
Control access to buckets using either a bucket ACL or using bucket policies

\item
By Default buckets are private and all objects stored inside them are private

\end{itemize}

\section{S3 Versioning - Exam Tips}
\begin{itemize}
\item
Stores all versions of an object (including all writes and even if you delete an object)

\item
Great backup tool

\item
Once enabled, versioning cannot be disabled, only suspended

\item
Integrates with lifecycle rules

\item
Versioning's MFA Delete capability, which uses multi-factor authentication, can be used to provide an additional layer of security
\end{itemize}

\section{S3 - Cross Region Replication Exam Tips}
\begin{itemize}
\item
Versioning must be enabled on both the source and destination buckets

\item
Regions must be unique

\item
Files in an existing bucket are not replicated automatically. All the subsequent updated files will be replicated automatically

\item
You cannot replicate to multiple buckets or use daisy chaining (at this time)

\item
Delete markers are replicated

\item
Deleting individual versions or delete markers will not be replicated

\item
Understand what cross region replication is at high level

\end{itemize}

\section{S3 Lifecycle Management Lab}
\begin{itemize}

\item
Can be used in conjunction with versioning

\item
Can be applied to current versions and previous versions

\item
Following actions can now be done:
	\begin{itemize}
	\item
	Transition to the standard -infrequent access storage class (minimum file size should be 128 KB and 30 days after the creation date)
	
	\item
	Archive to the glacier storage class (30 days after IA, if relevant)
	
	\item
	Permanently Delete
	
	\end{itemize}

\end{itemize}

\section{Cloudfront}

\textbf{What is a CDN?} \\
A Content Delivery Network (CDN) is a system of distributed servers (network) that delivers webpages and other web content to a user based on the geographic locations of the user, the origin of the webpage and a content delivery server

\begin{itemize}
\item
\textbf{Edge Location} - This is the location where the content will be cached. This is seperate to an AWS Region/AZ

\item
\textbf{Origin} - This is the origin of all the files that the CDN will distribute. This can be either an S3 bucker, an EC2 instance, an elastic load balancer or Route53

\item
\textbf{Distribution} This is the name given the CDN consists of a collection of Edge locations
\end{itemize}

\textbf{What is CloudFront} \\
Amazon CloudFront can be used to deliver your entire website, including dynamic, static, streaming, and interactive content using a global network of edge locations.Requests for your content are automatically routed to the nearest edge location, so content is delivered with best possible performance. Amazon CloudFront is optimized to work with other Amazon Web Services, like Amazon Simple Storage Service (Amazon S3), Amazon Elastic Compute Cloud (Amazon EC2), Amazon Elastic Load Balancing, and Amazon Route 53. Amazon CloudFront also works seamlessly with any non AWS origin server, which stors the original, definitive versions of your files. \\


\textbf{Key Terminology} \\
\begin{itemize}
\item
Web Distribution - Typically used for websites

\item
RTMP - Used for media streaming

\item
Edge locations are not just READ only, you can write to them too.(i.e. put an object on to them)

\item
Objects are cached for the life of the TTL

\item
You can clear cached objects, but you will be charged.

\end{itemize}

\section{S3 - Security and Encryption}
\begin{itemize}
\item
By default, all newly created buckets are PRIVATE

\item
You can setup acess control to your buckets using:
	\begin{itemize}
	\item
	Bucket Policies
	
	\item
	Access Control Lists
	\end{itemize}

\item
S3 buckets can be configured to create access logs. This can be done to another bucket

\item
Four different methods of encryption in S3:
	\begin{itemize}
	\item
	In Transit - SSL/TLS
	
	\item
	At Rest
		\begin{itemize}
		\item
		Server Side Encryption
			\begin{itemize}
			\item
			S3 Managed Keys - \textbf{SSE-S3}
			
			\item
			AWS Key Management Service, MAnaged Keys - \textbf{SSE-KMS}
			
			\item
			Server Side Encryption with customer provided keys - \textbf{SSE-C}
			\end{itemize}
			
		\item
		Client side Encryption
	
		\end{itemize}
	\end{itemize}
\end{itemize}

\section{Storage Gateway}

AWS Storage Gateway is a service that connects an on-premise software appliance with cloud-based storage to provide seamless and secure integration between an oragnization's on-premise IT environment and AWS's storage infrastructure. The service enables you to securely store date to the AWS cloud for scalable and cost effective storage.\\

\noindent
AWS Storage Gateway's software appliance is available for download as a virtual machine VM image that you install on a host in your datacenter. Storage Gateway supports either VMWare ESXi or Microsoft Hyper-V. Once you have installed your gatewat and associated it with your AWS account through the activation process, you can use the AWS Management Console to create the storage gateway option that is right for you.\\

Four Types of Storage Gateways: 
\begin{itemize}
	\item
	File Gateway (NFS) \textcolor{red}{New}
	
	\item
	Volumes Gateway (iSCSI)
	\begin{itemize}
		\item
		Stored Volumes	\textcolor{red}{Gateway Stored Volumes}
		
		\item
		Cached Volumes  \textcolor{red}{Gateway Cached Volumes}
	\end{itemize}
	
	\item
	Tape Gateway (VTL) \textcolor{red}{Gateway Virtual Tape Library}
\end{itemize}

\subsection{Gateway}
Files are stored as objects in your S3 buckets, accessed through a Network File System (NFS) mount point. Ownership, permission, and timestamps are durably stored in S3 in the user-metadata of the object associated with the file. Once the objects are transferred to S3, they can be managed as native S3 objects, and bucket policies such has versioning, lifecycle management, and cross-region replication apply directly to objects stored in your bucket. \\


\subsection{Volume Gateway}
The volume interface presents your applications with disk volumes using the iSCSI block protocol. Data written to these volumes can be asynchronously backed up as point in time snapshots of your volumes, and stored in the cloud as Amazon EBS (Elastic Block Store) snapshots. Snapshots are incremental backups that capture only changed blocks. All snapshot storage is also compressed to minimized your storage charges.\\

\noindent
\textbf{Stored Volumes}\\
\noindent
Stored Volumes let you store your primary data locally, while asynchronouslt backing up that data to AWs. Stored volumes provide your on-primies applications with lo latency access to their entire datasets, while providing durable, off-site backups. You can create storage volumes and mount them as iSCSI devices from your on premises application servers. Data written to your stored volumes is stored on your on-premises storage hardware. This data is asynchronously backed up to Amazon Simple Storage Service (Amazon S3) in the form of Amazon Elastic Block Store (Amazon EBS) snapshots. 1 GB - 16TB in size of Stored Volumes. \\

\noindent
\textbf{Cached Volumes}\\
\noindent
Cached volumes let you use Amazon Simple Storage Service (S3) as your primary data storage while retaining frequently accessed data locally in your storage gateway. Cached volumes minimize the need to scale your on-premises storage infrastructure, while still providing your applications with low latency access to their freuently accessed data. You can create storage volumes up to 32 TB in size and attach to them as iSCSI devices from your on-premises aplication servers. Your gateway stores data that you write to these volumes in Amazon S3 and retains recently read data in your on-premises storage gateway's cache and upload buffer storage. 1GB -32TB in size for cached volumes. \\

\noindent
\textbf{Tape Gateway}\\
\noindent
Tape Gateway offers a durable, cost-effective solution to archive your data in the AWS cloud. The VTL interface it provides lets you leverage your existing tape-based backup application infrastructure to store your date on virtual tape cartridges that you create on your tape gateway. Each tape gateway is preconfigured with a media changer and tape drives, which are available to your existing client backup applications as iSCSCI devices. You add tape cartridges as you need to archive your data. Supported by NetBackup, Backup Exec, Veeam etc. \\

\subsection*{Storage Gateway - Exam Tips}
\begin{itemize}
\item
\textbf{File Gateway} - For flat files, stored directly on S3.
\item
\textbf{Volume Gateway}
	\begin{itemize}
	\item
	Stored Volumes - Entire dataset is stored on site and is asynchronously backed up to S3

	\item
	Cached Volumes - Entire dataset is stored on S3 and the most frequently access data is cached on site
	\end{itemize}
\item
\textbf{Gateway Virtual Tape Library (VTL)} - Used for backup and uses popular backup applications like NetBackup, Backup Exec, Veeam etc.
\end{itemize}


\section{Snowball - Import Export Disc}
AWS Import/Export Disk accelerates moving large amounts of data into and out of the AWS cloud using portable storage devices for transport. AWS Import/Export disk transfers your data directly onto and off of storage devices using Amazon's high speed internal network and bypassing the internet. \\

\textbf{Types of Snowballs}
\begin{itemize}

\item
Snowball

\item
Snowball Edge

\item
Snowmobile
\end{itemize}

\subsection{Snowball}
Snowball is a petabyte-scale data transport solution that uses secure appliances to transfer large amounts of data into and out of AWS. Using Snowball addresses common challenges with large scale data transfers including high network costs, lon transfer times and security concerns. Transferring data with snowall is simple, fast ans secure and ca n e as little as one-fifth the cost of high speed internet.

80 TB snowball in all regions. Snowball uses multiple layers of security designed to protect your data including tamper-resistant enclosures, 256-bit encryption, and an industry-standard Trusted Platform Module (TPM) designed to ensure both security and full chain of custody of your data. Once the data transfer job has been processed and verified, AWS performs a software erasure of the Snowball appliance.

\subsection{Snowball Edge}
AWS Snowball Edge is a 100TB data transfer device with on board storage and compute capabilities. You can use Snowball Edge to move large amounts of data into and out of AWS, as a temporary storage tier for large local datasets, or to support local workloads in remote or offline locations.

Snowball Edge connects to your existing applications and infrastructure using standard storage interfaces, streamlining the data transfer process and miniizing setup and integration. Snowball Edge can cluster together to form a local storage tier and processs your data on-premises, helping ensure your applications continue to run even when they are not able to access the cloud.

\subsection{Snowmobile}
AWS Snowmobile is an exabyte scale data trsnfer service used to move extremely large amounts of data to AWS. You can transfer up to 100 PB per snowmobile, a 45-foot long ruggedized shipping container, pulled by a semi trailer truck. Snowmobile makes it eas to move massive volumes of data to the cloud, including video libraries, image repositories, or even a completed data center migration. Transferring data with snowmobile is secure, fast and cost effective.

\subsection*{Exam Tips}
\begin{itemize}
\item
Understand what snowball is

\item
Understand what import export is

\item
Snowball can do import to S3 and export from S3

\section{S3 Transfer Acceleration}
S3 Transfer Acceleration utilises the CloudFront Edge Network to accelerate your uploads to S3. Instead of uploading directly to your S3 bucket, you can use a distinct URL to upload directly to an edge location which will then transfer that file to S3. 

\end{itemize}
\section{Notes}
\begin{itemize}
\item
Read the S3 FAQs before taking the exam. It comes up a lot.

\item
Understand the difference between a region, an availability zone(AZ) and an edge location
	\begin{itemize}
	\item
	A region is a physical location in the world which sonsists of two or more Availability Zones (AZs)

	\item
	An AZ is one or more discrete data centers each with redundandt power, networking and connectivity, housed in seperate facilities

	\item
	Edge locations are endpoints for AWS which are used for caching content. Typically this consists of CloudFront, Amazon's Content Delivery Network (CDN)
\end{itemize}

\item
Remember that S3 is object based i.e. allows you to upload files

\item
Files can be from 0 bytes to 5TB

\item
There is unlimited storage

\item
Files are stored in buckets

\item
S3 is a universal namespace, i.e. names must be unique globally.

\item
Bucket URL looks like : https://s3-eu-west-1.amazonaws.com/acloudguru

\item
Read after write consistency for PUTS of new objects

\item
Eventual Consistency for overwrite PUTS and DELETES (can take some time to propagate)

\item
S3 Storage Classes/Tiers
	\begin{itemize}
	\item
	S3 - Durable, Immediately Available, Frequently Accessed)
	
	\item
	S3 -IA (durable, imediately availablem infrequently accessed)
	
	\item
	S3 - Reduced Redundancy Storage (data that is easily reproducible, such as thumbnails etc)
	
	\item
	Glacier - Archived Data, where you can wait 3-5  hours before accessing
	\end{itemize}

\item
Remember the core fundamentals of S3
	\begin{itemize}
	\item
	Key(name)
	
	\item	
	Value(data)
	
	\item
	Version ID
	
	\item
	Metadata
	
	\item
	Access Control Lists
	\end{itemize}

\item
Object based storage

\item
Not suitable to install an operating system on

\item
Stores all versions of an object (including all writes and even if you delete an object)

\item
Versioning is great backup tool

\item
Once enabled, versioning cannot be disabled, only suspended

\item
Versionng integrates with lifecycle rules

\item
Versioning's MFA Delete capability, which uses multi factor authentication, can be used to provide an additional layer of security

\item
Cross Region Replication, requires versioning enabled on source bucket as well as destination bucket

\item
Lifecycle management can be used in conjunction with versioning

\item
Lifecycle management can be applied to current and previoud versions

\item
Following actions can now be done :
	\begin{itemize}
	\item
	Transition to the standard : Infrequent access storage class (128 KB and 30 days after the creation date)
	
	\item
	Archive to the glacier storage class (30 days after IA, if relevant)
	
	\item
	Permanently Delete
	\end{itemize}

\item
CloudFront Edge Location - This is the location where content will be cached. This is seperate to an AWS Region/AZ

\item
Origin - This is the origin of all the files that CDN will distribute. This can be either an S3 Bucket, an EC2 instance, an elastic load balancer or Route 53

\item
Distribution - This is the name given the CDN which consists of a collection of Edge Locations
	\begin{itemize}
	\item
	Web Distribution - Typically used for websites
	
	\item
	RTMP - Used for Media Streaming
	\end{itemize}
	
\item
Edge locations are not just READ only, you can write to them too. (i.e. put an objects on to them)

\item
Objects are cached for the life of TTL

\item
You can clear cached objects, but you will be charged.

\item
By default, all newly created buckets are PRIVATE

\item
You can setup access controls to your buckets using:
	\begin{itemize}
	\item
	Bucket Policies
	
	\item
	Access Control Lists
	\end{itemize}

\item
S3 Buckets can be configured to create access logs which log all requests made to the S3 Bucket. This can be done to another bucket.

\item
Encryption (Read encryption section)

\item
Storage Gateway (Read Storage Gateway section)

\item
You can load files to S3 much faster by enabling multipart upload

\item
I can have 100 S3 buckets per account by default

\item
Read S3 FAQ before taking the exam. It comes up a LOT!
\end{itemize}


\end{document}