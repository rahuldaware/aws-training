\documentclass{article}
\usepackage{hyperref}
\usepackage{color}
\addtolength{\oddsidemargin}{-.875in}
\addtolength{\evensidemargin}{-.875in}
\addtolength{\textwidth}{1.75in}
\addtolength{\topmargin}{-.875in}
\addtolength{\textheight}{1.75in}
\pagestyle{plain}
\author{Rahul Daware}
\begin{document}
\title{EC2}
\maketitle
\newpage
\tableofcontents
\newpage

\section{EC2 101}
Amazon Elastic Compute Cloud (Amazon EC2) is a web service that provides resizable compute capacity in the cloud. Amazon EC2 reduces the time required to obtain and boot new server instances to minutes, allowing you to quickly scale capacity, both up and down, as your computing requirements change.

EC2 Pricing Models:
\begin{itemize}
\item
On Demand - Allows you to pay a fixed rate by the hour (or by the second) with no commitment

\item
Reserved - Provides you with a capacity reservation, and offe a significant discount on the hourly charge for an instance. Contract Terms are 1 Year or 3 Year Terms.

\item
Spot - Enables you to bid whatever price you want for instance capacity, providing for even great savings if your applications have flexible start and end times.

\item
Dedicated Hosts - Physical EC2 Server dedicated for your suse. Dedicated Hosts can help you reduce costs by allowing  you to use your existing server-bound software licenses.
\end{itemize}

\subsection{On Demand Pricing}
\begin{itemize}
\item
Users that want the low cost and flexiblility of Amazon EC2 without any up-front payment or long term commitment

\item
Applications with short term, spiky, or unpredictable workloads that cannot be interrupted.

\item
Applications being developed or tested on Amazon EC2 for the first time.
\end{itemize}

\subsection{Reserved Pricing}
\begin{itemize}
\item
Applications with steady state or predictable usage

\item
Applications that require reserved capacity

\item
Users able to make upfront payments to reduce their total computing costs even further

\item
\textbf{Standard Reserved Instances} - These offer up to 75\% off on demand instances. The more you pay up front and the longer the contract, the greater the discount.

\item
\textbf{Convertible Reserved Instances} - These offer up to 54\% off on demand capability to change the attrbutes of the RI as long as the exchange results in the creation of Reserved Instances of equal or greater value.

\item
\textbf{Scheduled Reserved Instances} - These are available to launch within the time windows tou reserve. This option allows you to match your capacity reservation to a predictable recurring schedule that only requires a fraction of a day, a week, or a month.
\end{itemize}

\subsection{Spot Pricing}
\begin{itemize}
\item
Applications that have flexible start and end times

\item
Applications that are only feasible at very low compute prices

\item
Users with urgent computing needs for large amounts of additional capacity
\end{itemize}

\subsection{Dedicated Hosts Pricing}
\begin{itemize}
\item
Useful for regulatory requirements that may not support multi-tenant virtualization

\item
Great for licensing which does not support multi-tenancy or cloud deployments

\item
Can be purchased On-Demand (hourly)

\item
Can be purchased as a reservation for up to 70\% off the on-demand price
\end{itemize}

\subsection{EC2 Instance Types}
	\begin{center}
		\begin{tabular}{ |p{1.5cm}|p{5cm}|p{8cm}|}		
		\hline
		\textbf{Family} & \textbf{Specialty} & \textbf{Use Case} \\
	 	\hline
	 	F1 & Field Programmable Gate Array & Genomics research. financial analytics, real-time video processing, big data etc \\ \hline
	 	I3 & High Speed Storage & NoSQL DBs, Data Warehousing etc \\ \hline
	 	G3 & Graphics Intensive & Video Encoding, 3D Application Streaming \\\hline
	 	H1 & High Disk Throughput & MapReduce based workloads, distributed file systems such as HDFS and MapR-FS \\ \hline
	 	T3 & Lowest Cost, General Purpose & Web Servers/Small DBs \\ \hline
	 	D2 & Dense Storage & Fileservers/Data Warehousing/ Hadoop \\ \hline
	 	R5 & Memory Optimized & Memory Intensive Apps/ DBs \\ \hline
	 	M5 & General Purpose & Application Servers \\ \hline
	 	C5 & Compute Optimized & CPU Intensive Apps/ DBs \\ \hline
	 	P3 & Graphics/ General Purpose GPU & Machine Learning , Bit Coing Mining etc \\ \hline
	 	X1 & Memory Optimized & SAP HANA/Apache Spark etc \\ \hline
	 	Z1D & High compute capacity and a high memory footprint & Ideal for electronic design automation (EDA) and certain relational database workloads with high per-core licensing costs. \\ \hline
	 	A1 & Arm-based workloads & Scale out workloads such as web servers \\ \hline
	 	U-6tb1 & Bare Metal & Bare Metal Capabilities that eliminate virtualization overhead \\	 	
	 	\hline
		\end{tabular}
	\end{center}
\subsection{Exam Tips}
\begin{itemize}
\item
Termination Protection is turned off by default, you must turn it on.

\item
On an EBS-back instance, the default action is for the root EBS volume to be deleted when the instance is terminated.

\item
EBS Root Volumes of your DEFFAULT AMIs cannot be encrypted. You can use a third party tol (such as bit locker etc) to encrypt the root volume, or this can be done when creating AMIs in the AWS console or using the API.

\item
Additional volumes can be encrypted.
\end{itemize}

\section{Security Groups Basics}
\begin{itemize}
\item
All inbound traffic is blocked by default.

\item
All outbound traffic is allowed.

\item
Changes to security groups take effect immediately.

\item
You can have any number of EC2 instances within a security group.

\item
You can have multiple security groups attached to EC2 instances.

\item
Security Groups are STATEFUL.

\item
If you create an inbound rule allowing traffic in, that traffic is automatically allowed back out again.

\item
You cannot block specific IP addresses using security groups, instead use network access control lists.

\item
You can specify allow rules, but no deny rules

\end{itemize}

\section{EBS}
Amazon Elastic Block Store (EBS) provides persistent block storage volumes for use with Amazon EC2 instances in the AWS cloud. Each Amazon EBS volume is automatically replicated within its availability zone to protect you from component failure, offering high availability and durability

Types of EBS Storage:
\begin{itemize}
\item
General purpose(SSD)

\item
Provisioned IOPS (SSD)

\item
Throughput optimised hard disk drive

\item
Cold hard disk drive

\item
Magnetic
\end{itemize}
\textbf{Solid State Drives:} 
\begin{center}
		\begin{tabular}{ |p{1.7cm}|p{6cm}|p{2cm}|p{1.5cm}|p{1.5cm}|p{1.5cm}|}
		\hline
		\textbf{Volume Type} & \textbf{Description} & \textbf{Use Cases} & \textbf{API Name} & \textbf{Volume Size} & \textbf{Max IOPS} \\ 
		\hline
		General Purpose SSD & General Purpose SSD volume that balances price and performance for wide variety of transactional workloads & Most Work Loads & gp2 & 1GiB-16TiB & 16,000 \\ \hline
		Provisioned IOPS SSED & highest-performance SSD volume designed for mission critical applications & Databases & io1 & 4GiB-16TiB & 64,000 \\ \hline
		\end{tabular}
\end{center}


\textbf{Hard Disk Drives: }
\begin{center}
		\begin{tabular}{ |p{1.7cm}|p{6cm}|p{2cm}|p{1.5cm}|p{1.5cm}|p{1.5cm}|}
		\hline
		\textbf{Volume Type} & \textbf{Description} & \textbf{Use Cases} & \textbf{API Name} & \textbf{Volume Size} & \textbf{Max IOPS} \\ 
		\hline
		Throughput Optimized HDD & Low cost HDD volume designed for frequently accessed, throughput intensive workloads & Big Data, Data Warehouses & st1 & 500GiB-16TiB & 500 \\ \hline
		Cold HDD & Lowest cost HDD volume designed for less frequently accessed workloads & File servers & sc1 & 500GiB-16TiB & 250 \\ \hline
		EBS Magnetic & Previous Generation HDD & Workloads where data is infrequently accessed & Standard & 1GiB-1TiB & 40-200 \\ \hline
		\end{tabular}
\end{center}

\subsection{Exam Tips}
\begin{itemize}
\item
Volumes exist on EBS. Think of EBS as a virtual hard disk.

\item
Snapshots exist on S3. Think of snapshots as a photograph of the disk.

\item
Snapshots are point in time copies of volumes.

\item
Snapshots are incremental - this means that only the blocks that have changed since your last snapshot are moved to S3.

\item
If it is your first snapshot, it may take some time.

\item
To create a snapshot for Amazon EBS volumes that serve as root devices, you should stop the instance before taking the snapshot.

\item
However you can take a snap while the instance is running.

\item
You can create AMIs from both volumes and snapshots.

\item
You can change EBS volume sizes on the fly, inclusing changing the size and storage type.

\item
Volumes will ALWAYS be in the same availability zone as the EC2 instance.

\item
To move an EC2 volumes from one AZ to another, take a snapshot of it, create an AMI from the snapshot and then use the AMI to launch the EC2 instance in a new AZ.

\item
To move an EC2 volume from one region to another, take a snapshot of it, create an AMI from the snapshot, and then copy the AMI from one region to the other. Then use the copied AMI to launch the new EC2 instance in the new region.
\end{itemize}

\section{AMI Types (EBS vs Instance Store)}
You can select your AMI based on:
\begin{itemize}
\item
Region (see Regions and AZ)

\item
Operating System

\item
Architecture (32 bit or 64 bit)

\item
Launch Permissions

\item
Storage for the root device (Root Device Volume)
	\begin{itemize}
	\item
	Instance Store (EPHEMERAL STORAGE)
	
	\item
	EBS Backed Volumes
	\end{itemize}

\end{itemize}
All AMIs are categorized as either backed by Amazon EBS or backed by instance store. For EBS volumes, the root device for an instance launched from the AMI is an Amazon EBS volume created from an Amazon EBS snapshot. For instance store volumes, the root device for an instance launched from the AMI is an instance store volume created from a remplate stored in Amazon S3.

\subsection{Exam Tips}
\begin{itemize}
\item
Instance Store Volumes are sometimes called Ephemeral Storage.

\item
Instance store volumes cannot be stopped. If the underlying host fails, you will lose your data.

\item
EBS backed instances can be stopped. You will not lose the data on this instance if it is stopped.

\item
You can reboot both, you will not lose your data.

\item
By default, both ROOT volumes will be deleted on termination. However, with EBS volumes, you can tell AWS to keep the root device volume.
\end{itemize}

\section{Volumes and Snapshots}
Steps to create encrypted volumes:
\begin{itemize}
\item
Create a snapshot of the unencrypted root device volume

\item
Create a copy of the snapshot and select the encrypt option

\item
Create an AMI from the encrypted snapshot

\item
Use that AMI to launch new encrypted instances
\end{itemize}
\subsection{Exam Tips}
\begin{itemize}
\item
Snapshots of encrypted volumes are encrypted automatically.

\item
Volumes restored from encrypted snaphosts are encrypted automatically.

\item
You can share snapshots, but only if they are unencrypted.

\item
These snapshots can be share with other AWS accounts or made public.
\end{itemize}

\section{CloudWatch 101}
Amazon CloudWatch is monitoring service to monitor your AWS resources, as well as the applications that you run on AWS. CloudWatch can monitor things like compute (EC2 instances, autoscaling groups, elastic load balancers, Route53 health checks), Storage and Content Delivery (EBS volumes, Storage Gateways, CloudFront).

\subsection{CloudWatch and EC2}
Host level metrics consists of:
\begin{itemize}
\item
CPU

\item
Network

\item
Disk

\item
Status Check
\end{itemize}

\subsection{What is AWS CloudTrail}
AWS CloudTrail increased visiblity into your user and resource activity by recording AWS management console actions and API calls. You can identify which users and accounts called AWS, the source IP address from which the calls were made, and when the calls occurred.

\subsection{Exam Tips}
\begin{itemize}
\item
CloudWatch is used for monitoring performance.

\item
CloudWatch can monitor most of AWS as well as your applications that run on AWS.

\item
CloudWatch with EC2 will monitor events every 5 minutes by default.

\item
You can create 1 minute intervals by turning on detailed monitoring.

\item
You can create CloudWatch alarms which trigger notifications.

\item
CloudWatch is all about performance. CloudTrail is all about auditing.

\item
Standard Monitoring is 5 minutes

\item
Detailed Monitoring is 1 minute

\item
You can create dashboards to see what is happening with your AWS environment

\item
you can set alarms that notify you when particular thresholds are hit.

\item
CloudWatch events helps you to respond to state changes in your AWS resources.

\item
CloudWatch logs helps you to aggregate, monitor and store logs.
\end{itemize}

\end{document}