\documentclass{report}
\usepackage{hyperref}
\author{Rahul Daware}
\begin{document}
\title{Chapter 4: S3}
\maketitle

\section*{What is S3?}
S3 provides developers and IT teams with secure, durable, highly scalable object storage. Amazon S3 is easy to use, with a simple web services interface to store and retrieve any amount of data from anywhere on the web. S3 is a safe place to store your files. It is object based storage. The date is spread across multiple devices and facilities.

\section*{S3 - The Basics}
\begin{itemize}
\item
S3 is object based - i.e. allows you to upload files

\item
Files can be from 0 bytes to 5 TB.

\item
There is unlimited storage.

\item
Files are stored in buckets. (Buckets are folder in cloud. Each bucket has universal namespace. Each bucket will have a DNS address. Bucket names have to unique.

\item
S3 is a universal namespace. i.e. names must be unique globally.

\item
URL look like this : https://s3-eu-west-1.amazonaws.com/acloudguru

\item
When you upload a file to S3, you will receive a HTTP 200 if the upload was successful

\item
Built for 99.99\% availability for the S3 platform

\item
Amazon guarantee 99.9\% availability

\item
Amazon guarantees 11 9s durability for S3 information

\item
Tiered Storage available

\item
Lifecycle Management

\item
Versioning

\item
Encryption

\item
Secure your data using Access Control Lists and Bucket Policies

\end{itemize}

\section*{Data Consistency Model for S3}
\begin{itemize}
\item
Read after write consistency for PUTS of new objects

\item
Eventual consistency for overwrite PUTS and DELETES (can take some time to propogate)
\end{itemize}

\section*{S3 is a simple key-value store}
S3 is object based. Objects consists of the following:
\begin{itemize}
\item
Key (This is simply name of the object
\item
Value ((This is simply the data and is made up of a sequence of bytes
\item
Version ID (Important for versioning)
\item
Metadata (Data about data you are storing)
\item
Subresources
	\begin{itemize}
	\item
	Access Control Lists
	
	\item
	Torrent
	
	\end{itemize}
\end{itemize}

\section*{S3 - Storage Tiers/Classes}
\begin{itemize}
\item
\textbf{S3 Standard} : 99.99\% availability 11 9s durability, stored redundantly across multiple facilities, and is designed to sustain the loss of 2 facilities concurrently.

\item
\textbf{S3 -IA} : (Infrequently Accessed): For data that is accessed less frequently, but requires rapid access when needed. Lower fee than S3, but you are charged a retrieval fee.

\item
\textbf{S3 One Zone - IA} Want a lower cost option for infrequently accessed data, but do not require the multiple availability zone data resilience.

\item
\textbf{Glacier} Very cheap, but used for archival only. Expedited, standard or bulk. A standard retrievel time takes 3-5 hours.
\end{itemize}

\section*{S3 - Charges}
Charged for:
\begin{itemize}
\item
Storage (Per GB basis)

\item
Requests (No. of requests)

\item
Storage Management Pricing ( Charged for tags on data)

\item
Data Transfer Pricing (Transferring data from one region to other)

\item
Transfer Acceleration : Amazon S3 Transfer acceleration enables fast, easy and secure transfers of files over long distances between your end users and an S3 bucket. Transfer acceleration takes advantage of Amazon CloudFront's globally distributed edge locations. AS the data arrives at an edge location, data is routed to Amazon S3 over an optimized network path.
\end{itemize}

\section*{S3 Exam Tips for S3 101}
\begin{itemize}
\item
Remember that S3 is object based. i.e. allows you to upload files

\item
Files can be from 0 bytes to 5 TB

\item
There is unlimited storage.

\item
Files are stored in buckets

\item
S3 is a universal namespace. That is names must be unique globally

\item
Read afte write consistency for PUTS of new objects

\item
Eventual Consistency for overwrite PUTS and DELETES (can take some time to propogate)

\item
Storage Classes/Tiers:
	\begin{itemize}
	\item
	S3 (durable, immediately available, frequently accessed)
	
	\item
	S3 -IA (Durable, immediately available, infrequently accessed)
	
	\item
	S3 One Zone - IA (event chaeaper than IA, but onle in one AZ)
	
	\item
	Glacier - Achived data, where you cn wait 3-5hours before accessing
	\end{itemize}

\item
Remember core fundamentals of S3 object - Key (name), Value (data) , Version ID, Metadata, Subresources (ACL, Torrent)

\item
Object based storage only (for files)

\item
Not suitable to install an operating system on

\item
Successful uploads give HTTP 200 response

\end{itemize}

\section*{Create an S3 Bucket - Exam Tips}
\begin{itemize}
\item
Buckets are a univeral name space

\item
Upload an object to S3 receive a HTTP 200 code

\item
S3, S3 -IA, S3 - One Zone IA, S3 Reduced Redundancy Storage

\item
Encryption
	\begin{itemize}
	\item
	Client Side Encryption
	
	\item
	Server Side Encryption with Amazon S3 Managed Keys (SSE-S3), Server Side Encryption with KMS (SSE-KMS), Server Side Encryption with Customer provided keys (SSE-C)
	\end{itemize}

\item
Control access to buckets using either a bucket ACL or using bucket policies

\item
By Default buckets are private and all objects stored inside them are private

\end{itemize}

\section*{S3 Versioning - Exam Tips}
\begin{itemize}
\item
Stores all versions of an object (including all writes and even if you delete an object)

\item
Great backup tool

\item
Once enabled, versioning cannot be disabled, only suspended

\item
Integrates with lifecycle rules

\item
Versioning's MFA Delete capability, which uses multi-factor authentication, can be used to provide an additional layer of security
\end{itemize}

\section*{S3 - Cross Region Replication Exam Tips}
\begin{itemize}
\item
Versioning must be enabled on both the source and destination buckets

\item
Regions must be unique

\item
Files in an existing bucket are not replicated automatically. All the subsequent updated files will be replicated automatically

\item
You cannot replicate to multiple buckets or use daisy chaining (at this time)

\item
Delete markers are replicated

\item
Deleting individual versions or delete markers will not be replicated

\item
Understand what cross region replication is at high level

\end{itemize}

\section*{S3 Lifecycle Management Lab}
\begin{itemize}

\item
Can be used in conjunction with versioning

\item
Can be applied to current versions and previous versions

\item
Following actions can now be done:
	\begin{itemize}
	\item
	Transition to the standard -infrequent access storage class (minimum file size should be 128 KB and 30 days after the creation date)
	
	\item
	Archive to the glacier storage class (30 days after IA, if relevant)
	
	\item
	Permanently Delete
	
	\end{itemize}

\end{itemize}

\section*{Notes}
\begin{itemize}
\item
Read the S3 FAQs before taking the exam. It comes up a lot.



\end{itemize}
\end{document}