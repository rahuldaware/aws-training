\documentclass{report}
\usepackage{hyperref}
\author{Rahul Daware}
\begin{document}
\title{Chapter 4: S3}
\maketitle

\section*{What is S3?}
S3 provides developers and IT teams with secure, durable, highly scalable object storage. Amazon S3 is easy to use, with a simple web services interface to store and retrieve any amount of data from anywhere on the web. S3 is a safe place to store your files. It is object based storage. The date is spread across multiple devices and facilities.

\section*{S3 - The Basics}
\begin{itemize}
\item
S3 is object based - i.e. allows you to upload files

\item
Files can be from 0 bytes to 5 TB.

\item
There is unlimited storage.

\item
Files are stored in buckets. (Buckets are folder in cloud. Each bucket has universal namespace. Each bucket will have a DNS address. Bucket names have to unique.

\item
S3 is a universal namespace. i.e. names must be unique globally.

\item
URL look like this : https://s3-eu-west-1.amazonaws.com/acloudguru

\item
When you upload a file to S3, you will receive a HTTP 200 if the upload was successful

\end{itemize}

\section*{Data Consistency Model for S3}
\begin{itemize}
\item
Read after write consistency for PUTS of new objects

\item
Eventual consistency for overwrite PUTS and DELETES (can take some time to propogate)
\end{itemize}

\section*{S3 is a simple key-value store}
S3 is object based. Objects consists of the following:
\begin{itemize}
\item
Key (This is simply name of the object
\end{itemize}
\end{document}