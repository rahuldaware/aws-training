\documentclass{article}
\usepackage{hyperref}
\usepackage{color}
\addtolength{\oddsidemargin}{-.875in}
\addtolength{\evensidemargin}{-.875in}
\addtolength{\textwidth}{1.75in}
\addtolength{\topmargin}{-.875in}
\addtolength{\textheight}{1.75in}
\pagestyle{plain}
\author{Rahul Daware}
\begin{document}
\title{Chapter 2: Identity Access Management (IAM)}
\maketitle
\newpage
\tableofcontents
\newpage


IAM allows you to manage users and their level of access to the AWS console 

\section{What does IAM gives you?}
\begin{itemize}
\item
Centralized control of your AWS account

\item
Shared access to your AWS account

\item
Granular permissions

\item
Identity Federation (including Active Directory, Facebook, LinkedIn etc)

\item
Multifactor Authentication

\item
Provide temporary access to users/devices when necessary

\item
Allows you to setup your own password rotation policy

\item
Integrates with many AWS services

\item
Supports PCI DSS Compliance
\end{itemize}

\section{Critical Terms}
\begin{itemize}
\item
Users - End Users (think people)

\item
Group - A collection of users under once set of permissions

\item
Roles - You create roles and assign them to AWS resources

\item
Policies - A document that defines one or more permissions

\end{itemize}

\section{Summary}
IAM consists of the following:
\begin{itemize}

\item
Users

\item
Groups (A way to group our users and apply policies to them collectively)

\item
Roles

\item
Policy Documents (Made of JSON)

\item
IAM is universal. It does not apply to regions at this time

\item
Root account is simply the account created when first setup your AWS account. It has complete Admin access.

\item 
New users are assigned \textbf{Access Key ID and Secret Access Keys} when first created

\item 
These are not the same as password, and you cannot use the access key ID and secret access key to login to the console. You can use this to access AWS via the APIs and command line however

\item
Always setup multifactor authentication on your root account

\item
You can create and customise your own password rotation policies

\end{itemize}



\end{document}