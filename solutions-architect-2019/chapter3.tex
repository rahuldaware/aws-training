\documentclass{article}
\usepackage{hyperref}
\usepackage{color}
\addtolength{\oddsidemargin}{-.875in}
\addtolength{\evensidemargin}{-.875in}
\addtolength{\textwidth}{1.75in}
\addtolength{\topmargin}{-.875in}
\addtolength{\textheight}{1.75in}
\pagestyle{plain}
\author{Rahul Daware}
\begin{document}
\title{Identity Access Management and S3}
\maketitle
\newpage
\tableofcontents
\newpage

\section{Identity Access Management 101}

IAM allows you to manage users and their level of access to the AWS console. It is important to understand IAM and how it works, both for the exam and for administering a company's AWS account in real life.

\subsection{Features of IAM}
\begin{itemize}
\item
Centralised control of your AWS account

\item
Shared access to your AWS account

\item
Granular permissions

\item
Identity Federation (including Active Directory, Facebook, Linkedin etc)

\item
Multifactor authentication

\item
Provide temporary access for users/devices and services where necessary

\item
Allows you to set up your own password rotation policy

\item
Integrates with many different AWS services

\item
Supports PCI DSS Compliance
\end{itemize}

\subsection{Key Terminology for IAM}
\begin{itemize}
\item
\textbf{Users} - End users such as people, employees of an organization etc.

\item
\textbf{Groups} - A collection of users. Each user in the group will inherit permissions of the group.

\item
\textbf{Policies} - Policies are made up of documents, called policy documents. These documents are in a format called JSON and they five permissions as to what a user/group/role is able to do.

\item
\textbf{Roles} - You create roles and then assign them to AWS resources
\end{itemize}

\section{What is S3?}

S3 provides developers and IT teams with secure, durable, highly-scalable object storage. Amazon S3 is easy to use, with a simple web services interface to store and retrieve any amount of data from anywhere on the web.

	\subsection{Features of S3} 
	\begin{itemize}
	\item
	S3 is a universal namespace. That is, names must be unique globally.
	
	\item
	When you upload a file to S3, you will receive HTTP 200 code
	
	\item
	S3 is Object based. Think of objects just as files. Objects consist of the 		following:
		\begin{itemize}
			\item
			Key (this is simply the name of the object)
			
			\item
			Value (This is simply the data and is made up of a sequence of 					bytes)
			
			\item
			Version ID (Important for versioning)
			
			\item
			Metadata (Data about data you are storing)
			
			\item
			Subresources - Access Control Lists, Torrents
		\end{itemize}
	
	\item
	Tiered Storage Available
	
	\item
	Lifecycle Management
	
	\item
	Versioning
	
	\item
	Encryption
	
	\item
	MFA Delete
	
	\item
	Secure your data using \textbf{Access Control Lists} and \textbf{Bucket 		Policies}
	\end{itemize}

	\subsection{How does data consistency work for S3?}
	\begin{itemize}
	\item
	Read after write consistency for PUTS of new objects
	
	\item
	Eventual consistency for overwrite PUTS and DELETES (can take some time to 		propagate)
	\end{itemize}
	
	\subsection{S3 - Guarantees}
	\begin{itemize}
	\item
	Build for 99.99\% availability ofr the S3 platform.
	
	\item
	Amazon guarantee 99.9\% availability.
	
	\item
	Amazon guarantees 99.999999999\% durability for S3 information. (Remember 11 x 9s)
	\end{itemize}
	
	\subsection{S3 Storage Classes}
	\begin{itemize}
	\item
	\textbf{S3 Standard} : 99.99\% availability, 99.99999999999\% durability, stored redundantly 		across multiple devices in multiple facilities, and is designed to sustain the loss of 2 	facilities concurrently.
	
	\item
	\textbf{S3 - IA(infrequetly Accessed)}: For data that is accessed less frequently, but requires rapid access when needed. Lower fee than S3, but you are charged a retrieval fee.
	
	\item
	\textbf{S3 One Zone - IA}: For where you want a lower-cost option for infrequently accessed data, but do not require the multiple availability zone data resilience.
	
	\item
	\textbf{S3 - Intelligent Tiering}: Designed to optimize costs by automatically moving data to the most cost-effective access tier, without performance impact or operational overhead.
	
	\item
	\textbf{S3 Glacier}: S3 Glacier is a secure, durable and low-cost storage class for data archiving. You can reliably store any amount of data at costs that are competitive with or cheaper than on-premises solutions. Retrieval times configurable from minutes to hours.
	
	\item
	\textbf{S3 Glacier Deep Archive}: S3 Glacier Deep Archive is Amazon S3's lowest cost storage class where a retrieval time of 12 hours is acceptable.
	\end{itemize}
	
	\subsection{S3- Charges}
	\begin{itemize}
	\item
	Storage
	
	\item
	Requests
	
	\item
	Storage Management Pricing
	
	\item
	Data Transfer Pricing
	
	\item
	Transfer Acceleration
	
	\item
	Cross Region Replication Pricing
	\end{itemize}
	
	\subsection{S3 Transfer Acceleration}
	Amazon S3 Transfer Acceleration enables fast, easy and secure transfers of files over long distances between your end users and an S3 bucket. Transfer Acceleration takes advantage of Amazon CloudFront's globally distributed edge locations. AS the data arrives at an edge location, data is routed to Amazon S3 over an optimized network path.
	
	\subsection{S3 Security and Encryption}
	By default, all newly created buckets are PRIVATE. You can setup access control to your buckets using Bucket Policies and Access Control Lists. S3 Buckets can be configured to create access logs which log all requests made to the S3 bucket. This can be sent to another bucket and even another bucket in another account.\textbf{Encryption in Transit} is achieved by SSL/TLS. \textbf{Encryption at Rest (Server Side)} is achieved by:
	\begin{itemize}
	\item
	S3 Managed Keys - SSE-S3, AWS
	
	\item
	AWS Key Management Service, Managed Keys-SSE-KMS
	
	\item
	Server Side Encryption with Customer Provided Keys-SSE-C
	\end{itemize}	 
	You can also encrypt on client side before sending to S3.
	\subsection{S3 -Versioning}
	\begin{itemize}
	\item
	Stores all versions of an object (including all writes and even if you delete an object)
	
	\item
	Great Backup Tool
	
	\item
	Once enabled, \textbf{versioning cannot be disabled}, only suspended
	
	\item
	Integrates ith \textbf{Lifecycle} Rules
	
	\item
	Versioning's \textbf{MFA Delete} capability, which uses multi-factor authentication, can be used to provide an additional layer of security.
	\end{itemize}
	
	\subsection{S3 Lifecycle Management}
	\begin{itemize}
	\item
	Automates moving your objects between the different storage tiers.
	
	\item
	Can be used in conjunction with versioning.
	
	\item
	Can be applied to current versions and previous versions.
	\end{itemize}
	
	\subsection{S3 Cross Region Replication}
	\begin{itemize}
	\item
	Versioning must be enabled on both the source and destination buckets.
	
	\item
	Regions must be unique.
	
	\item
	Files in an existing bucket are not replicated automatically.
	
	\item
	All subsequent updated files will be replicated automatically.
	
	\item
	Delete markers are not replicated.
	
	\item
	Deleting individual versions or deleting delete markers will not be replicated.
	\end{itemize}
	
	\subsection{S3 Transfer Acceleration}
	S3 Transfer Acceleration utilises the CloudFront Edge NEtwork to accelerate your uploads to S3. Instead of uploading directly to your S3 bucket, you can use a distinct URL to upload directly to an edge location which will then transfer that file to S3. You will get a distinct URL to upload to in the format <bucket-name>.s3-accelerate.aws.amazon.com
	
	\subsection{CloudFront}
	A content delivery network (CDN) is a system of distributed servers (network) that deliver webpages and other web content to a user based on the geographic locations of the user, the origin of the webpage, and a content delivery server.
	
	\begin{itemize}
	\item
	\textbf{Edge Location} - This is the location where content will be cached. This is separate to an AWS Region/AZ
	
	\item
	\textbf{Origin} - This is the origin of all the files that the CDN will distribute. This can be an S3 bucket, an EC2 instance, an elastic load balancer, or Route53.
	
	\item
	\textbf{Distribution} - This is the name given to the CDN which consists of a collection of edge locations.
	\end{itemize}
	
	Amazon CloudFront can be used to deliver your entire website, including dynamic, static, streaming, and interactive content using a global network of edge locations. Tequests for your content are automatically routed to the nearest edge location, so content is delivered with the best possible perfomance.Types of CloudFront Distributions: 
	\begin{itemize}
	\item
	Web Distribution - Typically used for websites
	
	\item
	RTMP - Used for media streaming
	\end{itemize}
	
	\subsection{Snowball}
	Snowball is a petabyte-scale data transport solution that uses secure appliances to transfer large amounts of data into and out of AWS. USing Snowball addresses common challenges with large-scale data transfers including high network costs, long transfer times, and security concerns.  Transferring data with snowball is simple, fast, secure and can be as little as one-fifth the cost of high speed internet. Snowball comes in either a 50TB or 80TB size. Snowball uses multiple layers of security designed to protect your data including tamper-resistant enclosures, 256-bit encryption, and an industry standard trusted platform module (TPM) designed to ensure boths security and full chain of custody of your data. Once the data transfer job has been processed and verfified, AWS performs a software erasure of the Snowball appliance. AWS Snowball Edge is a 100TB data transfer device with ob-board storage and compute capabilities. You can use Snowball Edge to move lage amounts of data into and out of AWS, as a temporary storage tier for large local datasets, or to support local workloads in remote or offline locations.  Snowball Edge connects to your existing applications and infrastructure using standard storage interfaces, streamlining the data transfer process and minimizing setup and integration. Snowball Edge can cluster together to form a local storage tier and process your data on-premises, helping ensure your applications continue to run even when they are not able to access the cloud. AWS Snowmobile is a Exabyte-scale data transfer service used to move extremely large amounts of data to AWS. You can transfer up to 100PB per snowmobile, a 45-foot long ruggedizds shipping container, pulled by a semi-trailer truck. Snowmobile makes it easy to move massive volumes of data to the cloud, including video libraries, image repositories, or even a complete data center migration. Transferring data with Snowmobile is secure, fast and cost effective.
	
	\begin{center}
		\begin{tabular}{ |p{3cm}|p{3cm}|p{3cm}|}		
		\hline
		Available Internet Connection & Theoretical Min. No. of days to transfer 100TB at 80\% network utilization & When to consider AWS Import/Export Snowball? \\
	 	\hline
	 	T3(44.736 Mbps) & 269 days & 2TB or more \\ 
	 	100Mbps & 120 days & 5TB or more \\ 
	 	1000Mbps & 12 days & 60TB or more \\ 
	 	\hline
		\end{tabular}
	\end{center}
	
	\subsection{Storage Gateway}
	AWS Storage Gateway is a service that connects an on-premises software appliance with cloud-based storage to provide seamless and secure integration between an organization's on-premise IT environment and AWS's storage infrastructure. The service enables you to securely store data to the AWS cloud for scalable and cost effective storage. AWS storage gateway's software appliance is available for you to download as a virtual machine (VM) image that you install on a host in your datacenter. Storage Gateway supports either VMWare ESXi or Microsoft Hyper-V. Once you have installed your gateway and associated it with your AWS account through the activate process, you can use the AWS Management Console to create the storage gateway option that is right for you. The three different types of Storage Gateway are as follows:
	\begin{itemize}
	\item
	File Gateway (NFS)
	
	\item
	Volume Gateway (iSCSI)
		\begin{itemize}
		\item
		Stored Volumes
		
		\item
		Cached Volumes
		\end{itemize}
		
	\item
	Tape Gateway
	\end{itemize}
		\subsubsection{File Gateway}
		Files are stored as objects in your S3 buckets, accessed through a Network File System (NFS) mount point. Ownership, permissions, and timestamps are durable stored in S3 in the user-metadata of the object associated with the file. Once objects are transferred to S3, the ycan be managed as native S3 objects, and bucket policieis such as versioning, lifecycle management, and cross-region replication apply directly to objects stored in your bucket.
		\subsubsection{Volume Gateway}
		The volume interface presents your applications with disk volumes using the iSCSI block protocol. Data written to these volumes can be asynchronously backed as point-in-time snapshots of your volumes , and stored in the cloud as Amazon EBS snapshots. Snapshots are incremental backups that capture only changed blocks. All snapshot storage is also compressed to minimize your storage charges. \\
		
		Stored Volumes let you stored your primary data locally, while asynchronously backing up that data to AWS. Stored volumes provide your on-premises applications with low-latency access to their entire datasets, while providing durable, off-site backups. You can create storage volumes and mount them as iSCSI devies from your on-premises application dservers. Data written to your stored volumes is  stored on your on-premises storage hardware. This data is asynchronously backed up to Amazon Simple Storage Service (Amazon S3) in the form of Amazon Elastic Block Store (Amazon EBS) snapshots. 1GB -16TB in size for stored volumes. \\
		
		Cached volumes let you use Amazon Simple Storage Service (Amazon S3) as your primary data storage while retaining frequently accessed data locally in your storage gateway. Cached volumes minimize the need to scale your on-premises storage infrastructure, while still providing your applications with low-latency access to their frequently accessed data. You can create storage volumes up to 32TB in size and attach them as iSCSI devices from your on-premises application servers. Your gateway stored data that you write to these volumes in Amazon S3 and retains recently read data in your on-premises storage gateway's cache and upload bufer storage. ! GB-32TB in size for cached volumes.
		\subsubsection{Tape Gateway}
		Tape Gateway offers a durable, cost-effective solution to archive your data in the AWS Cloud. The VTL interface it provides lets you leverage your existing tape-based backup application infrastructure to store your data on virtual tape cartridges that you create on your tape gateway. Each tape gateway is preconfigured with a media changer and tape drives, which are available to your existing client backup applications as iSCSI devices. Your add tape cartridges as you need to archive your data. Supported by NetBackup, BackupExec, Veeam etc. 
		\subsubsection{Exam Tips}
		\begin{itemize}
		\item
		File Gateway - For flat files stored directly on S3
		
		\item
		Volume Gateway (Stored Volumes) - Entire Data se is stored on site and is asynchronously backedup to S3.
		
		\item
		Volume Gateway (Cached Volumes) - Entire dataset is stored on S3 and the most frequently accessed data is cached on site.
		
		\item
		Gateway Virtual Tape Library
		\end{itemize}
		
\end{document}